\documentclass{article}

% Language setting
% Replace `english' with e.g. `spanish' to change the document language
\usepackage[english]{babel}

% Set page size and margins
% Replace `letterpaper' with `a4paper' for UK/EU standard size
\usepackage[letterpaper,top=2cm,bottom=2cm,left=3cm,right=3cm,marginparwidth=1.75cm]{geometry}

% Useful packages
\usepackage{amsmath}
\usepackage{graphicx}
\usepackage[colorlinks=true, allcolors=blue]{hyperref}

% Project 1: Focuses on methods from Part I (Bayesian thinking for ML, GLMs, Bayesian generative methods, Perceptrons, Delta rule and its variants, theoretical foundations, ...and applications thereof )

\title{AML Project 1}
\author{Team: Martin Tazon, Martin de Peretti, Daniel Weronski}

\begin{document}
\maketitle

\begin{abstract}
Your abstract.
\end{abstract}

\section{Introduction}

% Problem description and dataset presentation. Enumerate models that will be used and strong and weak point.
% How do they model p(y)
% Is the model discriminative or generative?

In this project, we will tackle down the problem of binary classification by addressing it from several perspectives. The dataset provides information about loan approvals [REF], therfore the different models we will train will attempt to determine if a loan is accepted or not in different ways. In general, the Bayesina framework will be favored over the frequentist since we are not particularly interested in computational efficiency of point estimates but rather in the uncertainty of the predictions and more importantly how the data have been generated.
We will make use of both Discriminative and Generative methods, since this will allow us to learn the hard or most probably soft boundaries that separeate our two classes as well as modeling the distirbutions of the two classes. 

\subsection{The problem}
In this project we will 
\subsection{Dataset}
The dataset at

\subsubsection{Dataset preprocessing}

Enumarate and justify preprocessing steps

\section{Logistic Linear Regression}

\section{Naive Bayes Classifier}

\section{Perceptron or Support Vector Machine ¿?}



\end{document}
